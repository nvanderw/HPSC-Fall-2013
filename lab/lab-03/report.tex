\documentclass{article}
\usepackage{graphicx}
\title{CSCI 4576 Lab 3}
\author{Nick Vanderweit}

\begin{document}
\maketitle
\section*{Part A}
To measure the relative performance of storing matrices as text vs. HDF5 files,
I modified the two C++ programs to accept arbitrarily many matrix arguments,
and to print out timing information about each input/output file. I then ran
the tools on input matrices of sizes ranging from 1000x1000 to 7000x7000. The
output of running \emph{io\_hdf} and \emph{io\_txt} in sequence thus looks
like:

\begin{verbatim}
matrix.1000x1000.txt,0.346169
output.matrix.1000x1000.txt,0.639346
matrix.2000x2000.txt,1.37757
output.matrix.2000x2000.txt,2.52065
matrix.3000x3000.txt,2.71792
output.matrix.3000x3000.txt,5.36238
matrix.4000x4000.txt,5.07548
output.matrix.4000x4000.txt,9.33633
matrix.5000x5000.txt,7.79706
output.matrix.5000x5000.txt,14.1913
matrix.6000x6000.txt,11.1057
output.matrix.6000x6000.txt,20.4491
matrix.7000x7000.txt,14.8286
output.matrix.7000x7000.txt,28.0903
matrix.1000x1000.h5,0.00477004
output.matrix.1000x1000.h5,0.0737159
matrix.2000x2000.h5,0.0118351
output.matrix.2000x2000.h5,0.289914
matrix.3000x3000.h5,0.024266
output.matrix.3000x3000.h5,0.655424
matrix.4000x4000.h5,0.0426998
output.matrix.4000x4000.h5,1.15974
matrix.5000x5000.h5,0.065702
output.matrix.5000x5000.h5,1.81137
matrix.6000x6000.h5,0.0910809
output.matrix.6000x6000.h5,2.59894
\end{verbatim}

The first value is the name of the file, and the second value is the number of
seconds elapsed while reading from or writing to that file. Using a small
Python script to filter through and plot these results, we obtain:

\begin{figure}[h!]
\includegraphics[width=0.9\textwidth]{io_speed.png}
\end{figure}

Here we see that HDF5 is several orders of magnitude faster than reading/writing
ordinary text files. In fact, comparing the 6000x6000 matrices, we can see
from the results that reading is around 122 times faster, while writing is 7.9
times faster.
\end{document}
